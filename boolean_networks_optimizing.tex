\documentclass[a4paper,10pt]{article}
\usepackage[utf8]{inputenc}

%opening
\title{Boolean network optimization}
\author{Jáchym Barvínek}

\begin{document}

\maketitle

%\begin{abstract}
%\end{abstract}

\section{Introduction}
In this work, we propose a method for modifying a Boolean network
in order to make the networks attractors behave in a desired way
while minimizing the number of regulatory changes to the original
network. We apply this method to a Boolean network representing
the Fanconi Aneamia regulatory pathways proposed in \cite{rodriguez1}.
The authors collected observed regulatory effects from number of 
other works and compiled them into a Boolean network model while
inferring some additional regulatory effects in order to make the
network always converge to a single certain period-2 attractor,
which they deemed a necessary condition for the network to be 
considered reasonable. The inferred additional regulatory effects
probably appealed to the authors intuition. We propose an alternative
approach to this appealing to the Occam's razor principle instead: To
automatically compute a minimal set of additional regulatory effects
which makes the network satisfy the required conditions on the
behavior of its attractors.

\section{Outline of the algorithm}
For simplicity we assume that we require all starting states converge
to a single attractor, which was the case in our application.
However, it could be easily generalized to a more complicated behavior.
The basic idea of our method is actually very simple and could be summarized
as follows:

\begin{enumerate}
 \item Represent the Boolean network with a fixed number of nodes
 as as binary string of fixed length. (See details below.)
 \item Run a multicriterial optimization genetic algorithm on networks with the above binary representation.
 (In particular, we used NSGA-II \cite{nsga2}.) The optimization criteria are the following:
 \begin{enumerate}
  \item The Hamming distance from the original unmodified Boolean network. (Minimize.)
  \item The number of starting states that arrive in the desired attractor from a randomly selected set of starting states. (Maximize.)
  Note, that the total number of possible starting states is $2^n$ where $n$ is the number of network nodes. 
  We can thus think of maximizing this criterion as doing a relaxation to the original intractable constraint problem.
 \end{enumerate}
 \item From the last generation select the individual which minimizes the first criterion and verify that 
 all starting states eventually converge to the required attractor using an expensive, exhaustive search.
 In other words, we verify that the solutions of the relaxed problem are feasible for the original problem.
\end{enumerate}

\section{Binary representation of the Boolean network}
To represent an arbitrary Boolean function in a disjunctive normal form (DNF),
we would generally need an exponential number of terms. That would of course make the problem not tractable.
However, in practice, it can be often possible to represent each of the Boolean function
in the network's nodes by a $k$-term-DNF with some small $k$. In our application, $k = 4$ was sufficient:
Each of the formulas representing Boolean functions in \cite{rodriguez1} could be converted to DNF 
with at most 4 terms. Then we made $k = 4$ a fixed parameter for representation of any network the algorithm
would consider. Using this bound, we can represent each term (conjunction) in the DNF
by a boolean vector of length $2n$ indicating for each logical variable (node) if there is a positive literal for 
the variable in the term and also if there is a negative literal for the variable.
In this manner, we can represent the entire Boolean network by a binary vector of length $2kn^2$.

\section{Parameters of the genetic algorithm}
Care needs to be taken for details of the evolutionary algorithm configuration.
Not all configurations that we attempted produced any useful results.
We present a configuration that eventually produced very good results,
however the computation ran for four days on 24 CPUs. So the parameters presented here
are probably far from optimal performance-wise. 
The algorithm was implemented using the DEAP library \cite{deap}.

\subsection{Initialization}
We used a population of 1024 individuals.
A freshly initialized individual was initialized so that each bit of the representation
had initial independent probability $\frac{1}{16}$ to have value 1 and $\frac{15}{16}$ probability to have value 0.

\subsection{Mutation}
The used mutation operator was the following: Each of $k$ randomly selected bits in the representation
was flipped with probability $\frac{9}{10}$.

\subsection{Crossover}
The used crossover operator was the two-point crossover.
The crossover points were only on the boundaries of the DNF representations,
so a representation of one Boolean function always remains contiguous during crossover.
Crossover probability was $\frac{9}{10}$.
As mentioned earlier, the selection mechanism was NSGA-II \cite{nsga2}.

\section{Results}
We now briefly discuss the results of our application of this method.
In the supplementary materials for \cite{rodriguez1}, in Table 1,
there are 31 total regulatory effects marked as inferred by the authors.
When converted to the DNF representation defined above, the Hamming distance
from the network with these effects omitted is 64. 
Using the method desired above, we were able to breed four Boolean networks,
which had the Hamming distance criterion value only 3 and all of them
also succeeded in the exhaustive test of the attractor behavior.
This is a significant simplification of the regulatory changes 
compared to those proposed by the authors of \cite{rodriguez1}.
An interesting observation is that all of those individuals involved
negative self-regulation of the CHKREC node, which was also inferred 
by the authors of \cite{rodriguez1}.

\bibliography{boolean_networks_optimizing}{}
\bibliographystyle{plain}

\end{document}
